\documentclass[12pt,letter,titlepage]{article}
\linespread{1.5}
\usepackage[margin=1in]{geometry}
\usepackage[utf8]{inputenc}
\usepackage[english]{babel}
\usepackage{lipsum}

\usepackage{amsmath, amssymb, amsfonts, amsthm, fouriernc, mathtools}
% mathtools for: Aboxed (put box on last equation in align envirenment)
\usepackage{microtype} %improves the spacing between words and letters

\usepackage{graphicx}
\graphicspath{ {./pics/} {./eps/}}
\usepackage{epsfig}
\usepackage{epstopdf}

\usepackage{dirtytalk}
\usepackage{hyphenat}
\usepackage{siunitx}
\usepackage{ragged2e}

%%%%%%%%%%%%%%%%%%%%%%%%%%%%%%%%%%%%%%%%%%%%%%%%%%
%% COLOR DEFINITIONS
%%%%%%%%%%%%%%%%%%%%%%%%%%%%%%%%%%%%%%%%%%%%%%%%%%
\usepackage[svgnames]{xcolor} % Enabling mixing colors and color's call by 'svgnames'
%%%%%%%%%%%%%%%%%%%%%%%%%%%%%%%%%%%%%%%%%%%%%%%%%%
\definecolor{MyColor1}{rgb}{0.2,0.4,0.6} %mix personal color
\newcommand{\textb}{\color{Black} \usefont{OT1}{lmss}{m}{n}}
\newcommand{\blue}{\color{MyColor1} \usefont{OT1}{lmss}{m}{n}}
\newcommand{\blueb}{\color{MyColor1} \usefont{OT1}{lmss}{b}{n}}
\newcommand{\red}{\color{LightCoral} \usefont{OT1}{lmss}{m}{n}}
\newcommand{\green}{\color{Turquoise} \usefont{OT1}{lmss}{m}{n}}
%%%%%%%%%%%%%%%%%%%%%%%%%%%%%%%%%%%%%%%%%%%%%%%%%%




%%%%%%%%%%%%%%%%%%%%%%%%%%%%%%%%%%%%%%%%%%%%%%%%%%
%% FONTS AND COLORS
%%%%%%%%%%%%%%%%%%%%%%%%%%%%%%%%%%%%%%%%%%%%%%%%%%
%    SECTIONS
%%%%%%%%%%%%%%%%%%%%%%%%%%%%%%%%%%%%%%%%%%%%%%%%%%
\usepackage{titlesec}
\usepackage{sectsty}
%%%%%%%%%%%%%%%%%%%%%%%%
%set section/subsections HEADINGS font and color
\sectionfont{\color{MyColor1}}  % sets colour of sections
\subsectionfont{\color{MyColor1}}  % sets colour of sections

%set section enumerator to arabic number (see footnotes markings alternatives)
\renewcommand\thesection{\arabic{section}.} %define sections numbering
\renewcommand\thesubsection{\thesection\arabic{subsection}} %subsec.num.

%define new section style
\newcommand{\mysection}{
\titleformat{\section} [runin] {\usefont{OT1}{lmss}{b}{n}\color{MyColor1}} 
{\thesection} {3pt} {} } 

%%%%%%%%%%%%%%%%%%%%%%%%%%%%%%%%%%%%%%%%%%%%%%%%%%
%       CAPTIONS
%%%%%%%%%%%%%%%%%%%%%%%%%%%%%%%%%%%%%%%%%%%%%%%%%%
\usepackage{caption}
\usepackage{subcaption}
%%%%%%%%%%%%%%%%%%%%%%%%
\captionsetup[figure]{labelfont={color=Turquoise}}

%%%%%%%%%%%%%%%%%%%%%%%%%%%%%%%%%%%%%%%%%%%%%%%%%%
%       !!!EQUATION (ARRAY) --> USING ALIGN INSTEAD
%%%%%%%%%%%%%%%%%%%%%%%%%%%%%%%%%%%%%%%%%%%%%%%%%%
%using amsmath package to redefine eq. numeration (1.1, 1.2, ...) 
%%%%%%%%%%%%%%%%%%%%%%%%
\renewcommand{\theequation}{\thesection\arabic{equation}}

%set box background to grey in align environment 
\usepackage{etoolbox}% http://ctan.org/pkg/etoolbox
\makeatletter
\patchcmd{\@Aboxed}{\boxed{#1#2}}{\colorbox{black!15}{$#1#2$}}{}{}%
\patchcmd{\@boxed}{\boxed{#1#2}}{\colorbox{black!15}{$#1#2$}}{}{}%
\makeatother
%%%%%%%%%%%%%%%%%%%%%%%%%%%%%%%%%%%%%%%%%%%%%%%%%%




%%%%%%%%%%%%%%%%%%%%%%%%%%%%%%%%%%%%%%%%%%%%%%%%%%
%% DESIGN CIRCUITS
%%%%%%%%%%%%%%%%%%%%%%%%%%%%%%%%%%%%%%%%%%%%%%%%%%
\usepackage[siunitx, american, smartlabels, cute inductors, europeanvoltages]{circuitikz}
%%%%%%%%%%%%%%%%%%%%%%%%%%%%%%%%%%%%%%%%%%%%%%%%%%



\makeatletter
\let\reftagform@=\tagform@
\def\tagform@#1{\maketag@@@{(\ignorespaces\textcolor{red}{#1}\unskip\@@italiccorr)}}
\renewcommand{\eqref}[1]{\textup{\reftagform@{\ref{#1}}}}
\makeatother
\usepackage{hyperref}
\hypersetup{colorlinks=true}

\usepackage{pygmentize}
\usepackage{minted}
\usepackage{multicol}
\usepackage{graphicx}

%%%%%%%%%%%%%%%%%%%%%%%%%%%%%%%%%%%%%%%%%%%%%%%%%%
%% PREPARE TITLE
%%%%%%%%%%%%%%%%%%%%%%%%%%%%%%%%%%%%%%%%%%%%%%%%%%
\title{\blue CS 4414: Operating Systems\\
\blueb Machine Problem 3: File Management}
\author{Michael Eller}
\date{22 November 2016}
%%%%%%%%%%%%%%%%%%%%%%%%%%%%%%%%%%%%%%%%%%%%%%%%%%

\begin{document}
\maketitle

\noindent{I was successful in implementing all functions on the FAT 16 file system. For clarity the commands that the user can enter into my program are colored in {\color{orange}\textbf{\textit{orange}}}}, and all functions that are implemented in the underlying C++ code are colored in {\color{cyan}\textbf{\textit{cyan}}}.
\section*{Introduction}
The File Allocation Table (FAT) is a computer file system architecture that organizes and manages data clusters. The FAT system also displays the file structure in a human-readable format. The objective of this project was to explore the FAT 16 file architecture and implement a few select functions: {\color{orange}\textbf{\textit{cd}}} (change directory), {\color{orange}\textbf{\textit{ls}}} (list segments/files), {\color{orange}\textbf{\textit{cpout}}} (copy out), and {\color{orange}\textbf{\textit{cpin}}} (copy in).

\section*{Implementation}
The program operates as a simple shell that can traverse and manipulate a FAT 16 file system. The following functions are implemented in the C++ programming language, and all paths are relative to the user's present working directory. The program has one command-line parameter: the path to a FAT 16 raw data file.

\subsection*{cd <directory>}{
The {\color{orange}\textbf{\textit{cd}}} command is implemented with the combination of {\color{cyan}\textbf{\textit{change\_directory}}}() and \\{\color{cyan}\textbf{\textit{follow\_path}}}() functions. The {\color{cyan}\textbf{\textit{change\_directory}}}() function takes in one string as a parameter, and searches the user's current directory (all directory entries of the current cluster) and attempts to match the file name, which is the first eight bytes of each entry, to the string parameter. If a match is found, the function loads the first sector of the matched directory's starting cluster (bytes 26 and 27) into the user's current directory, which is a global variable. Since the {\color{cyan}\textbf{\textit{change\_directory}}}() function can only change directories by one folder at a time, the {\color{cyan}\textbf{\textit{follow\_path}}}() function is used in order to change multiple directories with one function call. It simply takes in a list of strings with all the directories and calls {\color{cyan}\textbf{\textit{change\_directory}}}() on each of them. All functions implemented in this program use {\color{cyan}\textbf{\textit{follow\_path}}}() in some manner.
}
\subsection*{ls [directory]}{
If no argument is given, {\color{orange}\textbf{\textit{ls}}} displays all of the files and folders in the user's present working directory. Otherwise, the function lists all files and folders from the path given by the user. After saving the user's current location in a temporary variable, {\color{cyan}\textbf{\textit{follow\_path}}}() is called in order to load the required directory. In the same manner that {\color{cyan}\textbf{\textit{change\_directory}}}() loops through to find a matching filename, {\color{cyan}\textbf{\textit{list\_directory}}}() loops through the appropriate directory and prints every file name that is not a long file name or the volume ID. The function then loads the starting location back into the user's current working directory.
}
\subsection*{cpout <src> <dest>}{
Between {\color{orange}\textbf{\textit{cpout}}} and {\color{orange}\textbf{\textit{cpin}}}, I found that {\color{orange}\textbf{\textit{cpout}}} is far easier to implement. The function begins by calling {\color{cyan}\textbf{\textit{follow\_path}}}() to the source file's directory. Then the FAT is consulted to check if the file is larger than a single cluster. As long as the active cluster is not the last cluster in the chain, the function continually reads source file data into a cluster-sized buffer and writes the data to the destination file on the computer's file system. The function also keeps track of the remaining bytes in the source file in order to only write the remaining bytes stored in the last cluster of the chain.
}
\subsection*{cpin <src> <dest>}{
The {\color{orange}\textbf{\textit{cpin}}}() function is very similar to {\color{orange}\textbf{\textit{cpout}}}(); however, instead of reading from the FAT, {\color{orange}\textbf{\textit{cpin}}}() finds the first open cluster in the FAT and starts writing source data there. If the file is larger than one cluster, {\color{orange}\textbf{\textit{cpin}}}() repeats this process while writing each successive cluster number to the index of the FAT that corresponds to the previous cluster in the chain. Additionally, {\color{orange}\textbf{\textit{cpin}}}() builds and writes the directory entry in the path specified by the destination argument.
}
\section*{Problems Encountered}
In general I found that I understood the homework conceptually, but was slowed down by little issues. For example, I created a text file on my computer using the \textit{touch} terminal command. Apparently the filename can appear as all uppercase, but the actual name required to open it is lowercase. This was extremely avoidable and easy to fix but slowed down my productivity since it took approximately 30 minutes to find. Besides small issues like that, the hardest part of this project was maintaining context. I used a global variable to hold the user's current location, and I had frequent issues using {\color{cyan}\textbf{\textit{lseek}}}() and returning to the correct location. The GDB debugger was extremely valuable in chasing down these issues. 

\noindent{\textbf{NOTE:} I assumed that a directory was not more than the size of one cluster, that directories did not need to be in alphabetical order, and that files can exist with same name.}

\section*{Testing and Analysis}
I primarily used a hex editor to verify any changes that I made to the sample disk file. Once I verified single level {\color{orange}\textbf{\textit{cd}}} and {\color{orange}\textbf{\textit{ls}}} I moved on to multiple level testing in both directions (forward vs. `..'). The {\color{orange}\textbf{\textit{cpout}}}() function was easy to verify since the result is a file on the user's computer. I used {\color{orange}\textbf{\textit{cpout}}}() to to verify {\color{orange}\textbf{\textit{cpin}}}() by copying out the same file I copied in. Similarly single and multiple level directory paths were tested as well as single and multiple cluster sized files.

\section*{Conclusion}
My implementation of the FAT 16 file manager is fully functional as specified by the problem description. I would like to acknowledge that I used the data structures provided by Professor Grimshaw and a significant amount of information gathered from \href{http://wiki.osdev.org/FAT#FAT_16}{\color{blue}\underline{this wiki}}.

\pagebreak
\section*{Code}
\linespread{1}
\subsection*{makefile}{
\scriptsize{
\begin{minted}{bash}
## Michael Eller mbe9a
## OS Machine Problem 3
## 22 November 2016
## makefile

OBJS = file_manager.o
CC = g++
DEBUG = -g
CFLAGS = -Wall -c $(DEBUG)
LFLAGS = -Wall $(DEBUG)

fat: $(OBJS)
	$(CC) $(LFLAGS) $(OBJS) -o fat

file_manager.o: file_manager.h file_manager.cpp
	$(CC) $(CFLAGS) file_manager.cpp

clean:
	\rm *.o *~ fat

tar:
	tar cfv mbe9a.tar fat file_manager.h file_manager.cpp makefile mbe9a.pdf
\end{minted}
}}

\subsection*{file\_manager.h}{
\scriptsize{
\begin{minted}{cpp}
// Michael Eller mbe9a
// OS Machine Problem 3
// 22 November 2016
// file_manager.h (structs and function protos)

#ifndef FILE_MANAGER_H
#define FILE_MANAGER_H

// entry attribute definitions
#define READ_ONLY	0x01
#define HIDDEN 		0x02
#define SYSTEM 		0x04
#define VOLUME_ID 	0x08
#define DIRECTORY 	0x10
#define ARCHIVE		0x20
#define LFN			0x0F
#include <sys/types.h>
#include <unistd.h>
#include <fcntl.h>
#include <stdio.h>
#include <stdlib.h>
#include <iostream>
#include <string.h>
#include <list>
#include <errno.h>
using namespace std;

// struct to hold the FAT16 boot sector
typedef struct {
    unsigned char jmp[3];
    char oem[8];
    unsigned short sector_size;
    unsigned char sectors_per_cluster;
    unsigned short reserved_sectors;
    unsigned char number_of_fats;
    unsigned short root_dir_entries;
    unsigned short total_sectors_short; // if zero, later field is used
    unsigned char media_descriptor;
    unsigned short fat_size_sectors;
    unsigned short sectors_per_track;
    unsigned short number_of_heads;
    unsigned int hidden_sectors;
    unsigned int total_sectors_long;
    
    unsigned char drive_number;
    unsigned char current_head;
    unsigned char boot_signature;
    unsigned int volume_id;
    char volume_label[11];
    char fs_type[8];
    char boot_code[448];
    unsigned short boot_sector_signature;
} __attribute((packed)) Fat16BootSector;
 
 // struct to hold a FAT16 entry
typedef struct {
    unsigned char filename[8];
    unsigned char ext[3];
    unsigned char attributes;
    unsigned char reserved[10];
    unsigned short modify_time;
    unsigned short modify_date;
    unsigned short starting_cluster;
    unsigned int file_size;
} __attribute((packed)) Fat16Entry;

// function to retrieve the FAT value of the active cluster
bool get_fat_value(unsigned short* value);

// function to return the offset of the first data sector of a given cluster
unsigned int first_sector_of_cluster(unsigned int cluster);

// read dir entries in single cluster (directory)
void list_directory(string path);

// change_directory helper function
void next_entry(unsigned int starting_sector, unsigned short offset);

// method to change the PWD string, another helper function for 'cd'
void update_pwd(unsigned int direction);

// implementation of cd command
Fat16Entry change_directory(string path, bool ls);

// helper function for main, takes in a char* and outputs a list of arguments
list<char*> build_arguments(char* token);

// helper function for main, takes in a list<char*>* and call change_directory
Fat16Entry follow_path(list<char*>* arguments, bool ls);

// helper function to test if we are in the root directory
void check_and_set_root();

// copy out function implementation
void copy_out(char* dest, list<char*>* arguments);

// helper function for copy in, returns the offset of the dir entry written by this function
// very similar to 'ls'
unsigned int write_dir_entry(Fat16Entry dir_entry, list<char*>* arguments);

// copy in function implementation, file modify data and time are neglected, attributes are assumed 0x00
void copy_in(char* name, unsigned int* src_file, list<char*>* arguments);

#endif
\end{minted}
}}

\subsection*{file\_manager.cpp}{
\scriptsize{
\begin{minted}{cpp}
// Michael Eller mbe9a
// OS Machine Problem 3
// 22 November 2016
// file_manager.cpp (main)

#include "file_manager.h"

// some globals
// holds the boot sector of the disk
Fat16BootSector BS;

// track state/position of file system
Fat16Entry CURRENT;

// maintain the current path to display for user
string PWD;

// entire volume
unsigned int DISK;

// where data starts on the disk
unsigned int first_data_sector;

// first root directory sector, root sector is treated differently
unsigned int first_root_dir_sector;

// are you in the root directory?
bool root;

// function to retrieve the FAT value at the active cluster (is it the last cluster in the chain?)
bool get_fat_value(unsigned short* value, bool first_cluster, unsigned short active_cluster)
{
	// save current location
	unsigned int location = lseek(DISK, 0, SEEK_CUR);

	unsigned char FAT_table[BS.sector_size];
	unsigned short fat_offset;
	if (first_cluster) fat_offset = CURRENT.starting_cluster * 2;
	else fat_offset = active_cluster * 2;
	unsigned short fat_sector = BS.reserved_sectors + (fat_offset / BS.sector_size);
	unsigned short ent_offset = fat_offset % BS.sector_size;

	// put the fat_sector into FAT_table
	lseek(DISK, fat_sector * BS.sector_size, SEEK_SET);
	read(DISK, &FAT_table, sizeof(FAT_table));
	*value = *(unsigned short*)&FAT_table[ent_offset];

	// return to original location
	lseek(DISK, location, SEEK_SET);

	// check if there are chained clusters
	if (*value >= 0xFFF8) return false;

	// not accounting for bad clusters here
	return true;
}
\end{minted}
\pagebreak
\begin{minted}{cpp}
// function to return the offset of the first data sector of a given cluster
unsigned int first_sector_of_cluster(unsigned int cluster, bool file)
{
	// don't calculate if the user is in the root directory
	if (root && !file) return first_root_dir_sector;
	// else calculate from the given cluster
	return ((cluster - 2) * BS.sectors_per_cluster) + first_data_sector;
}

// helper function to get file/directory name
void get_filename(char name[])
{
	// copy the filename into the char array
	memcpy(name, &CURRENT.filename, sizeof(CURRENT.filename));

	// append the file extension if it's not a directory
	if (!(CURRENT.attributes & DIRECTORY))
	{
		for (unsigned int i = 0; i < sizeof(CURRENT.filename); i++) {
			if (name[i] == ' ') //|| (i == sizeof(CURRENT.filename) - 1)
			{
				// if we are at the end of the filename, start appending at the next position
				//if ((i == sizeof(CURRENT.filename) - 1) && (name[i] != ' ')) i++;

				// add the dot
				name[i] = '.';

				// add the rest of the extension
				for (int j = 1; j <= 3; j++) name[i + j] = CURRENT.ext[j - 1];

				// add the null terminator
				name[i + 4] = '\0';
				return;
			}
		}
	}
	// get rid of extra spaces in directory names
	else for (unsigned int i = 0; i < sizeof(CURRENT.filename); i++) if (name[i] == ' ') name[i] = '\0';
}
\end{minted}
\pagebreak
\begin{minted}{cpp}
// read dir entries in single cluster (directory)
void list_directory()
{
	// start by reading in and saving the first sector of the active cluster
	unsigned short counter = 0;
	unsigned int first_sector = first_sector_of_cluster(CURRENT.starting_cluster, false);
	lseek(DISK, first_sector * BS.sector_size, SEEK_SET);
	read(DISK, &CURRENT, sizeof(Fat16Entry));

	// keep listing as long as the first byte != 0
	while (1)
	{
		// don't print it if it's a long file name or if it's the volume ID
		if ((CURRENT.attributes & LFN) || (CURRENT.attributes & VOLUME_ID))
		{
			// read next entry
			counter++;
			lseek(DISK, (first_sector * BS.sector_size) + (counter * sizeof(Fat16Entry)), SEEK_SET);
			read(DISK, &CURRENT, sizeof(Fat16Entry));
			continue;
		}
		else
		{
			// get the filename
			// initialize with max possible size
			char name[sizeof(CURRENT.filename) + sizeof(CURRENT.ext) + 2];
			get_filename(name);

			// display the name
			cout << name << endl;

			// read the next entry
			counter++;
			lseek(DISK, (first_sector * BS.sector_size) + (counter * sizeof(Fat16Entry)), SEEK_SET);
			read(DISK, &CURRENT, sizeof(Fat16Entry));
			
			// break if the first byte is 0
			if (CURRENT.filename[0] == 0x00) break;
		}
	}
	// return to the first sector
	lseek(DISK, first_sector * BS.sector_size, SEEK_SET);
	read(DISK, &CURRENT, sizeof(Fat16Entry));
}
\end{minted}
\pagebreak
\begin{minted}{cpp}
// change_directory helper function
void next_entry(unsigned int starting_sector, unsigned short offset)
{
	// load sector at given offset into CURRENT
	lseek(DISK, (starting_sector * BS.sector_size) + (offset * sizeof(Fat16Entry)), SEEK_SET);
	read(DISK, &CURRENT, sizeof(Fat16Entry));
}

// method to change the PWD string, another helper function for 'cd'
void update_pwd(unsigned int direction)
{
	// take one level off of PWD if the user did 'cd ..'
	if (direction == 0)
	{	
		// used to index the PWD string from the end, for figuring out where to stop deleting
		unsigned short index = 0;

		// get rid of all the end characters until first '/' is reached
		for (string::reverse_iterator str_riter = PWD.rbegin() + 1; str_riter < PWD.rend(); ++str_riter)
		{
			index++;
			if (*str_riter == '/') break;
		}
		// delete the last directory
		PWD = PWD.substr(0, strlen(PWD.c_str()) - index);
	}
	// add name of the current directory to PWD
	else
	{
		// get directory name
		char* name = strtok((char*)CURRENT.filename, " ");

		// append it to PWD
		PWD += name;

		// formatting
		PWD += '/';
	}
}
\end{minted}
\pagebreak
\begin{minted}{cpp}
// implementation of cd command
Fat16Entry change_directory(string path, bool ls)
{
	// get the cluster of pwd, before we start searching
	unsigned int starting_sector = first_sector_of_cluster(CURRENT.starting_cluster, false);

	// stay in current directory
	if (!strcmp(path.c_str(), ".")) next_entry(starting_sector, 0);

	// go back one directory
	else if (!strcmp(path.c_str(), "..")) 
	{
		next_entry(starting_sector, 1);
		// update the PWD string if we are actually cd'ing
		if (!ls) update_pwd(0);
		// check and see if we are now looking at the root directory
		check_and_set_root();
	}
	// search for next directory
	else
	{
		// offset to iterate through directory
		unsigned int offset = 2;
		if (root) offset = 1;
		char name[sizeof(CURRENT.filename) + sizeof(CURRENT.ext) + 2];
		while (1)
		{
			// start at the third entry in directory
			next_entry(starting_sector, offset);
			// stop if the name wasn't found and it's the end of the cluster
			if (CURRENT.filename[0] == 0x00)
			{
				cout << "No such directory." << endl;
				break;
			}
			// ignore long file names
			if (CURRENT.attributes & LFN)
			{
				offset++;
				continue;
			}
			// get file/directory name
			get_filename(name);

	  		// is it a directory and does it match?
	  		if ((CURRENT.attributes & DIRECTORY) && (!strcmp(path.c_str(), name)))
	  		{
	  			// go to the first sector of the starting cluster and update the path
	  			root = false;
				// update the path if this wasn't used in an 'ls' command
	  			if(!ls) update_pwd(1);
	  			// continue to the first sector of the entry's starting cluster
	  			next_entry(first_sector_of_cluster(CURRENT.starting_cluster, false), 0);
	  			return CURRENT;
	  		}
	  		// if it's a file
	  		else if ((!strcmp(path.c_str(), name))) 
	  		{
	  			// continue to the first sector of the entry's starting cluster
	  			return CURRENT;
	  		}

	  		// try the next entry
	  		offset++;
	  	}
	}
	return CURRENT;
}

// helper function for main, takes in a char* and outputs a list of arguments
list<char*> build_arguments(char* token)
{
	// parse first argument
	list<char*> arguments;

	// add all to a list and return it
	while (token != NULL) 
		{
			arguments.push_back(token);
			token = strtok(NULL," /");
		}
	return arguments;
}

// helper function for main, takes in a list<char*>* and call change_directory
Fat16Entry follow_path(list<char*>* arguments, bool ls)
{
	Fat16Entry file_params;

	// follow the path by calling change_directory for each directory in path
	while (!(arguments->empty()))
	{
		file_params = change_directory(arguments->front(), ls);
		arguments->pop_front();
	}
	return file_params;
}

// helper function to test if we are in the root directory
void check_and_set_root()
{
	if (CURRENT.starting_cluster == 0) root = true;
	else root = false;
}
\end{minted}
\pagebreak
\begin{minted}{cpp}
// copy out function implementation
void copy_out(char* dest, list<char*>* arguments)
{
	// create the file locally
	unsigned int output_file = open(dest, O_RDWR | O_CREAT | O_TRUNC, S_IRWXU);

	// find the fat entry
	Fat16Entry file_entry = follow_path(arguments, true);

	// save the file size
	unsigned int file_size = file_entry.file_size;

	// buffer to read in clusters
	unsigned int cluster_size = BS.sectors_per_cluster * BS.sector_size;
	char cluster_buffer[cluster_size];

	// go to the first cluster and read in the data
	lseek(DISK, first_sector_of_cluster(file_entry.starting_cluster, true) * BS.sector_size, SEEK_SET);
	read(DISK, &cluster_buffer, cluster_size);

	// check if there are more clusters
	unsigned short fat_value = 0;
	if (get_fat_value(&fat_value, true, 0))
	{
		// write to the output file and update the remaining file size
		write(output_file, &cluster_buffer, cluster_size);
		file_size -= cluster_size;

		// move to the next cluster
		lseek(DISK, first_sector_of_cluster(fat_value, true) * BS.sector_size, SEEK_SET);

		// pass by value and reference at once could cause issues
		unsigned short active_cluster = fat_value;
		while (get_fat_value(&fat_value, false, active_cluster))
		{
			read(DISK, &cluster_buffer, cluster_size);

			// write to the output file and update the remaining file size
			write(output_file, &cluster_buffer, cluster_size);
			file_size -= cluster_size;

			// check if there is another cluster
			if (file_size <= cluster_size) break;

			// else, keep going
			active_cluster = fat_value;

			// move to the next cluster
			lseek(DISK, first_sector_of_cluster(fat_value, true) * BS.sector_size, SEEK_SET);
		}
		// read one last time
		read(DISK, &cluster_buffer, cluster_size);
	}
	// write the remaining bytes to the output file
	write(output_file, &cluster_buffer, file_size);

	// close the file
	close(output_file);
}
\end{minted}
\pagebreak
\begin{minted}{cpp}
// helper function for copy in, returns the offset of the dir entry written by this function
// very similar to 'ls'
unsigned int write_dir_entry(Fat16Entry dir_entry, list<char*>* arguments)
{
	// go to the directory if not there already
	if (!arguments->empty()) follow_path(arguments, true);

	// get the offset of the the first sector in this directory
	unsigned int first_sector = first_sector_of_cluster(CURRENT.starting_cluster, false);

	// read until there isn't an entry
	unsigned short counter = 0;
	while(1)
	{
		// keep reading if it's a long file name or if it's the volume ID
		if ((CURRENT.filename[0] == 0x00) && !(CURRENT.attributes & LFN)) break;

		// read next entry
		counter++;
		lseek(DISK, (first_sector * BS.sector_size) + (counter * sizeof(Fat16Entry)), SEEK_SET);
		read(DISK, &CURRENT, sizeof(Fat16Entry));
	}
	// write the dir entry here
	lseek(DISK, (first_sector * BS.sector_size) + (counter * sizeof(Fat16Entry)), SEEK_SET);
	unsigned int dir_entry_offset = lseek(DISK, 0, SEEK_CUR);
	write(DISK, &dir_entry, sizeof(Fat16Entry));

	// return to the offset at the beginning of the method
	read(DISK, &CURRENT, sizeof(Fat16Entry));
	return dir_entry_offset;
}
\end{minted}
\pagebreak
\begin{minted}{cpp}
// copy in function implementation, file modify data and time are neglected, attributes are assumed 0x00
void copy_in(char* name, unsigned int* src_file, list<char*>* arguments)
{
	// get the cluster of pwd, before we start searching
	unsigned int starting_offset = lseek(DISK, 0, SEEK_CUR);

	// build the dir entry
	Fat16Entry dir_entry;
	dir_entry.attributes = 0;
	unsigned int file_size = lseek(*src_file, 0, SEEK_END);
	lseek(*src_file, 0, SEEK_SET);
	dir_entry.file_size = file_size;

	// build filename
	unsigned char filename[8];
	unsigned int i;
	for (i = 0; i < strlen(name) - 4; i++) filename[i] = name[i];

	// add spaces for any left over space
	while (i < 8) 
	{
		filename[i] = ' ';
		i++;
	}

	// copy into the struct
	memcpy(&dir_entry.filename, &filename, 8);

	// file extension
	unsigned short index = 2;
	for (i = strlen(name) - 1; i > strlen(name) - 4; i--) 
	{
		dir_entry.ext[index] = name[i];
		index--;
	}

	// read the entire FAT table
	unsigned char FAT_table[BS.fat_size_sectors * BS.sector_size];
	lseek(DISK, BS.reserved_sectors * BS.sector_size, SEEK_SET);
	read(DISK, &FAT_table, sizeof(FAT_table));

	// find clusters for the file data
	// start table index at 2 because 0 and 1 are reserved
	index = 2;
	unsigned int table_entry_offset;
	unsigned int prev_table_entry_index = 0xFFFF;
	bool found = false;
	unsigned char src_buffer[BS.sectors_per_cluster * BS.sector_size];
	while (1)
	{
		// get the value from the table
		unsigned short table_value = *(unsigned short*)&FAT_table[index * sizeof(short)];
\end{minted}
\pagebreak
\begin{minted}{cpp}
		if (table_value == 0)
		{
			// calculate the offset of this table entry
			table_entry_offset = (index * sizeof(short)) + (BS.reserved_sectors * BS.sector_size);

			// fill the fat table with the old index
			if (found)
			{
				lseek(DISK,(prev_table_entry_index * sizeof(short)) + 
					(BS.reserved_sectors * BS.sector_size), SEEK_SET);
				write(DISK, &index, sizeof(short));
			}
			// first cluster found
			else
			{
				// write FFFF in case this is the only cluster
				dir_entry.starting_cluster = index;
				found = true;
				lseek(DISK, table_entry_offset, SEEK_SET);
				write(DISK, &prev_table_entry_index, sizeof(short));
			}
			prev_table_entry_index = index;

			// start writing data to the cluster
			if (file_size > (BS.sectors_per_cluster * BS.sector_size))
			{
				read(*src_file, &src_buffer, BS.sectors_per_cluster * BS.sector_size);
				lseek(DISK, first_sector_of_cluster(index, true) * BS.sector_size, SEEK_SET);
				write(DISK, &src_buffer, BS.sectors_per_cluster * BS.sector_size);

				// decrement remaining bytes
				file_size -= BS.sectors_per_cluster * BS.sector_size;
			}
			else
			{
				// write remaining file data to the cluster
				read(*src_file, &src_buffer, file_size);
				lseek(DISK, first_sector_of_cluster(index, true) * BS.sector_size, SEEK_SET);
				write(DISK, &src_buffer, file_size);

				// write FFFF to the last cluster
				lseek(DISK, table_entry_offset, SEEK_SET);
				unsigned short last_entry = 0xFFFF;
				write(DISK, &last_entry, sizeof(short));
				break;
			}
		}
		index++;
	} 
	// find a spot for the file entry and write it
	lseek(DISK, starting_offset, SEEK_SET);
	write_dir_entry(dir_entry, arguments);

	// return to the original position
	lseek(DISK, starting_offset, SEEK_SET);
	read(DISK, &CURRENT, sizeof(Fat16Entry));
}
\end{minted}
\pagebreak
\begin{minted}{cpp}
// main function
int main(int argc, char *argv[])
{
	// open the sample FAT16 volume
	DISK = open(argv[1], O_RDWR);

	// read boot sector into struct
	read(DISK, &BS, sizeof(Fat16BootSector));

	// compute the size of the root directory
	unsigned int root_dir_sectors = ((BS.root_dir_entries * 32) + (BS.sector_size - 1)) / BS.sector_size;

	// compute the offset of the first data sector
	first_data_sector = BS.reserved_sectors + 
		(BS.number_of_fats * BS.fat_size_sectors) + root_dir_sectors;

	// compute the offset of the first root directory sector
	first_root_dir_sector = first_data_sector - root_dir_sectors;

	// set the current directory to the ROOT directory
	lseek(DISK, first_root_dir_sector * BS.sector_size, SEEK_SET);
	read(DISK, &CURRENT, sizeof(Fat16Entry));

	// put the root dir in path
	PWD.append("/");

	// char pointers to parse user input
	char *buf = (char*)malloc(sizeof(char)*1024);
	char *token;
	char *src;
	char *dest;

	// list to hold inputs for commands
	list<char*> arguments;

	while (1)
	{
		//root = true;
		check_and_set_root();

		// print the current directory
		cout << ":" << PWD << "> ";

		// read input from stdin
		fgets(buf, 1024, stdin);
		for (int x = 0; (unsigned)x < strlen(buf); x++) if ( buf[x] == '\n' || buf[x] == '\r' ) buf[x] = '\0';

		// exit program in user inputs 'exit'
		if (!strcmp(buf, "exit")) break;
\end{minted}
\pagebreak
\begin{minted}{cpp}
		// parse input
		else
		{
			// get command
			token = strtok(buf, " ");
			if (!strcmp(buf, "ls")) 
			{
				// parse argument
				token = strtok(NULL, " /");

				// list present directory
				if (token == NULL) list_directory();

				// list directory path
				else
				{
					// get the cluster of pwd, before we start searching
					unsigned int starting_sector = first_sector_of_cluster(CURRENT.starting_cluster,false);

					// put the path into a list
					arguments = build_arguments(token);

					// follow the path by calling change_directory for each directory in path
					follow_path(&arguments, true);

					// list that directory
					list_directory();

					// return to the original sector
					lseek(DISK, starting_sector * BS.sector_size, SEEK_SET);
					read(DISK, &CURRENT, sizeof(Fat16Entry));
				}
			}
			// change directory
			else if (!strcmp(buf, "cd"))
			{
				// parse argument, similar to 'ls'
				token = strtok(NULL, " /");

				// put the path into a list
				arguments = build_arguments(token);

				// change directory
				follow_path(&arguments, false);	
			}
			// copy out
			else if (!strcmp(buf, "cpout"))
			{
				// get the cluster of pwd, before we start searching
				unsigned int starting_sector = first_sector_of_cluster(CURRENT.starting_cluster, false);

				// parse src and dest
				src = strtok(NULL, " ");
				dest = strtok(NULL, " ");

				// put the path into a list
				arguments = build_arguments(strtok(src, " /"));

				// copy the file to the destination on computer
				copy_out(dest, &arguments);

				// return to the original sector
				lseek(DISK, starting_sector * BS.sector_size, SEEK_SET);
				read(DISK, &CURRENT, sizeof(Fat16Entry));
			}
\end{minted}
\pagebreak
\begin{minted}{cpp}
			// copy in
			else if (!strcmp(buf, "cpin"))
			{
				// get the cluster of pwd, before we start searching
				unsigned int starting_sector = first_sector_of_cluster(CURRENT.starting_cluster, false);

				// parse src and dest
				src = strtok(NULL, " ");
				dest = strtok(NULL, " ");

				// put the path into a list
				arguments = build_arguments(strtok(dest, " /"));

				// open the file specified by src on unified system
				unsigned int src_file = open(src, O_RDWR);

				// pop off the file name
				char* name = arguments.back();
				arguments.pop_back();

				// copy the file into dest path in the fat16 volume
				copy_in(name, &src_file, &arguments);

				close(src_file);

				// return to the original position
				lseek(DISK, starting_sector * BS.sector_size, SEEK_SET);
				read(DISK, &CURRENT, sizeof(Fat16Entry));
			}
		}
	}
	close(DISK);
	return 0;
}
\end{minted}
}}
\end{document}
